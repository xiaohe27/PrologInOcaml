There are several editions of prolog currently available online: SWI-prolog, GNU-prolog, Amzi-prolog, and so on. They are not identical. When we were doing the literature review, we found an implementation of Amzi-prolog by Ocaml, but Amzi-prolog is more of an old fashion and its documentation are no longer available online. So our implementation is more close to SWI-prolog due to the better availability of reference. Because we are implementing a subset of prolog, only a subset of tokens are defined.
The overall layout and structure of lexer and parser in this project is the same as our mp homework. In lexer, we firstly define useful regular expressions, and then match symbols or regular expressions with corresponding tokens: there are independent entries for multi-line comments, single and double quoted string contents. In parser, we firstly define language tokens used by lexer and parser, and then the "goal" nonterminal of our grammar: we allow program entry parsing (with query) and rules entry parsing (without parsing), and finally we construct the stratification needed for unambiguous parsing with reference to the precedence table we found on SWI-prolog manual.

