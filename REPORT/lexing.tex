Our implementation of prolog is more close to SWI-prolog due to the better availability of reference. Because we are implementing a subset of prolog, only a subset of tokens are defined. As an resource, we found an implementation of Amzi-prolog by Ocaml online\cite{gitcode}, but Amzi-prolog is more of an old fashion and its documentation are no longer available online. We just used a some of its regular expressions and token explanations, which are shared between different versions of prolog.
The overall layout and structure of lexer and parser in this project is the same as our mp homework. In lexer, we firstly define useful regular expressions, and then match symbols or regular expressions with corresponding tokens: there are independent entries for multi-line comments, single and double quoted string contents. In parser, we firstly define language tokens used by lexer and parser, and then the `goal' nonterminal of our grammar: we allow program entry parsing (with query) and rules entry parsing (without query), and finally we construct the stratification needed for unambiguous parsing with reference to the precedence table we found on SWI-prolog manual.

